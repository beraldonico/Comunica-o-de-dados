\documentclass{article}
\usepackage[utf8]{inputenc}
\usepackage[portuguese,brazilian]{babel}
\usepackage[top=2cm, bottom=2cm, left=4cm, right=4cm]{geometry}

\title{Trabalho 1 comunicação de dados}
\author{Nicolas Beraldo}
\date{Setembro 2018}

\usepackage{natbib}
\usepackage{graphicx}
\usepackage{indentfirst}
\usepackage{float}
\usepackage{multicol}
\usepackage{caption}

\begin{document}
    \maketitle
        \begin{center}
            Simulação de Emissor e Receptor com diferentes tipos de codificação e frequência de corte para um filtro passa-baixo.
        \end{center}
    \section*{Introdução}
       Neste trabalho iremos implementar, em Octave, como um emissor de sinais digitais ira converter o nível de tensão(0V ou 5V) usando um filtro passa-baixo com frequência de cortes diferentes e como um receptor interpreta o sinal filtrado. As frequências de corte utilizadas foram as harmônicas. Também simularemos um ruido para ver como o receptor interpreta o sinal afetado.

    \section{Parte 1}
        Nessa parte implementamos o básico de como converter uma stream de bits em uma amostra de tensão. 
    
        Utilizamos a stream 0101010111000 com uma amostragem de 100 pontos por bit e frequência de 1 Kbps. Utilizamos a convenção comum de tensão, se o bit é 0(zero) a tensão é 0V(zero) e se o bit é 1(um) a tensão é 5V(cinco). Obtivemos o seguinte gráfico:
        \begin{figure}[H]
            \begin{center}
                \includegraphics[scale=0.7]{Parte_1.png}
                \caption{Stream de bit convertido para neveis de tensão}
                \label{fig:Stream de bits convertido}
            \end{center}
        \end{figure}
        
    \section{Parte 2}
        Agora implementamos um filtro passa-baixo que utiliza algumas harmônicas como frequência de corte. Considerando a frequência da entrada como 1Kbps = 1000 bps e primeira harmônica sendo a metade da frequência de entrada os valores das frequências de corte são as seguintes:
        \begin{itemize}
            \item 1/2 frequência harmônica = 250 Hz
            \item 1 frequência harmônica = 500 Hz
            \item 3 frequência harmônica = 1500 Hz
            \item 5 frequência harmônica = 2500 Hz
            \item 7 frequência harmônica = 3500 Hz
            \item 9 frequência harmônica = 4500 Hz
        \end{itemize}
        Usamos o pacote "signal" do Octave que tem as funções "butter" e "filter" implementados. Essas funções são responsáveis por realizar o processo de filtragem do sinal. 
        
        Os gráficos a seguir comparam o sinal de entrada com o sinal apos passar no filtro:
        \begin{figure}[H]
            \begin{multicols}{2}
                \begin{figure}[H]
                    \begin{center}
                        \includegraphics[scale=0.45]{Parte_2_1.png}
                        \caption{1/2 harmônica}
                        \label{fig:Parte_2_1}
                    \end{center}
                \end{figure}
                \begin{figure}[H]
                    \begin{center}
                        \includegraphics[scale=0.45]{Parte_2_3.png}
                        \caption{3 harmônica}
                        \label{fig:Parte_2_3}
                    \end{center}
                \end{figure}
                
                \begin{figure}[H]
                    \begin{center}
                        \includegraphics[scale=0.45]{Parte_2_2.png}
                        \caption{1 harmônica}
                        \label{fig:Parte_2_2}
                    \end{center}
                \end{figure}
                \begin{figure}[H]
                    \begin{center}
                        \includegraphics[scale=0.45]{Parte_2_4.png}
                        \caption{5 harmônica}
                        \label{fig:Parte_2_4}
                    \end{center}
                \end{figure}
            \end{multicols}
        \end{figure}
        \begin{figure}[H]
            \begin{multicols}{2}
                \begin{figure}[H]
                    \begin{center}
                        \includegraphics[scale=0.45]{Parte_2_5.png}
                        \caption{7 harmônica}
                        \label{fig:Parte_2_5}
                    \end{center}
                \end{figure}
                \begin{figure}[H]
                    \begin{center}
                        \includegraphics[scale=0.45]{Parte_2_6.png}
                        \caption{9 harmônica}
                        \label{fig:Parte_2_6}
                    \end{center}
                \end{figure}
            \end{multicols}
            \caption{Sinal de entrada vs. Sinal filtrado}
            \label{fig:Parte_2}
        \end{figure}
        Após o sinal passar pelo filtro ele seria enviado para um receptor e o interpretaria para obter os mesmos dados de entrada. O receptor tem que estar em sincronia com o emissor, logo ambos possuem a informação da frequencial e a quantidade de amostrar por bit. Utilizando uma logica simples conseguimos fazer o receptor identificar os bits, nos deixando com os seguintes gráficos, eles comparam o sinal de entrada, o sinal filtrado e o sinal de saída:
        
        Em azul a entrada, em amarelo o sinal filtrado e em laranja o sinal de saída,
        \begin{figure}[H]
            \begin{multicols}{2}
                \begin{figure}[H]
                    \begin{center}
                        \includegraphics[scale=0.45]{Parte_2_7.png}
                        \caption{1/2 harmônica}
                        \label{fig:Parte_2_7}
                    \end{center}
                \end{figure}
                \begin{figure}[H]
                    \begin{center}
                        \includegraphics[scale=0.45]{Parte_2_8.png}
                        \caption{1 harmônica}
                        \label{fig:Parte_2_8}
                    \end{center}
                \end{figure}
            \end{multicols}
        \end{figure}
        \begin{figure}[H]
            \begin{multicols}{2}
                \begin{figure}[H]
                    \begin{center}
                        \includegraphics[scale=0.45]{Parte_2_9.png}
                        \caption{3 harmônica}
                        \label{fig:Parte_2_9}
                    \end{center}
                \end{figure}
                \begin{figure}[H]
                    \begin{center}
                        \includegraphics[scale=0.45]{Parte_2_11.png}
                        \caption{7 harmônica}
                        \label{fig:Parte_2_11}
                    \end{center}
                \end{figure}
                \begin{figure}[H]
                    \begin{center}
                        \includegraphics[scale=0.45]{Parte_2_10.png}
                        \caption{5 harmônica}
                        \label{fig:Parte_2_10}
                    \end{center}
                \end{figure}
                \begin{figure}[H]
                    \begin{center}
                        \includegraphics[scale=0.45]{Parte_2_12.png}
                        \caption{9 harmônica}
                        \label{fig:Parte_2_12}
                    \end{center}
                \end{figure}
            \end{multicols}
            %\captionsetup{figure = 2}
            \caption{Sinal de entrada vs. Sinal filtrado vs. Sinal de saída}
            \label{fig:Parte_21}
        \end{figure}
        Podemos perceber que nesse tipo de codificação a saída já pode ser obtida com a primeira harmônica mas com o tempo do bit muito atrasado. Podemos afirmar que o emissor e receptor tem que estar perfeitamente sincronizados para que a entrada e saída sejam idênticas pois na codificação normal não há um sinal para reiniciar o clock do receptor, que não foi implementado em nenhum código pois não era o foco do trabalho.
        
        Também é possível notar que o sinal de saída tem um tempo melhor a partir da terceira harmônica e aumentando a harmônica o tempo fica cada vez mais preciso, podemos ver isso na nona harmônica que o sinal de saída tem um tempo muito próximo ao de entrada.
        
        Abaixo temos a comparação entre a entrada e a saída. A variável "data out" é uma matriz de 6 linhas e 13 colunas pois as linhas são as saídas de cada harmônica:
        \begin{figure}[H]
            \begin{center}
                \includegraphics[scale=.5]{Saida_Emissor_receptor.png}
                \caption{comparação entre Emissor e receptor}
                \label{fig:Saida_Emissor_receptor}
            \end{center}
        \end{figure}
        Percebe-se que a partir da terceira harmônica a stream de saída é igual ao de entrada.
    \section{Parte 3}
        Nesta parte alteramos o código da parte anterior para adaptar ele para as codificações diferentes, as quais são NZR-I, Manchester e Manchester Diferencial. Cada codificação teu sua forma de criar e interpretar o sinal e cada uma tem as suas dificuldades de implementação mas ao mesmo tempo tem as vantagens, as quais não foram implementadas.
        
        \subsection{NZR-I}
            Nessa codificação o bit 0 mantêm a tensão do bit anterior e o bit 1 inverte a tensão do bit anterior, no nosso caso usamos as tensões 5V e -5V. Os gráficos a seguir mostram a comparação entre entrada, sinal filtrado e saída:
            \begin{figure}[H]
                \begin{multicols}{2}
                    \begin{figure}[H]
                        \begin{center}
                            \includegraphics[scale=0.45]{NZR_I_1.png}
                            \caption{1/2 harmônica}
                            \label{fig:NZR_I_1}
                        \end{center}
                    \end{figure}
                    \begin{figure}[H]
                        \begin{center}
                            \includegraphics[scale=0.45]{NZR_I_2.png}
                            \caption{1 harmônica}
                            \label{fig:NZR_I_2}
                        \end{center}
                    \end{figure}
                    \end{multicols}
            \end{figure}
            \begin{figure}[H]
                \begin{multicols}{2}
                    \begin{figure}[H]
                        \begin{center}
                            \includegraphics[scale=0.45]{NZR_I_3.png}
                            \caption{3 harmônica}
                            \label{fig:NZR_I_3}
                        \end{center}
                    \end{figure}
                    \begin{figure}[H]
                        \begin{center}
                            \includegraphics[scale=0.45]{NZR_I_5.png}
                            \caption{7 harmônica}
                            \label{fig:NZR_I_5}
                        \end{center}
                    \end{figure}
                
                    \begin{figure}[H]
                        \begin{center}
                            \includegraphics[scale=0.45]{NZR_I_4.png}
                            \caption{5 harmônica}
                            \label{fig:NZR_I_4}
                        \end{center}
                    \end{figure}
                    \begin{figure}[H]
                        \begin{center}
                            \includegraphics[scale=0.45]{NZR_I_6.png}
                            \caption{9 harmônica}
                            \label{fig:NZR_I_6}
                        \end{center}
                    \end{figure}
                \end{multicols}
                %\captionsetup{figure = 2}
                \caption{Sinal de entrada vs. Sinal filtrado vs. Sinal de saída de NZR-I}
                \label{fig:NZR_I}
            \end{figure}
            Nesse caso percebemos que a comunicação foi eficaz na quinta harmônica e por causa da implementação do código do receptor os tempos são sempre exatos pois ele trabalha com a amostragem dos bit e os seleciona baseado na diferença entre o elemento 50 do bit e do elemento 50 do bit anterior.
            
            É possível perceber que a saída é igual a entrada a partir da quinta harmônica.
            \begin{figure}[H]
                \begin{center}
                    \includegraphics[scale=0.5]{Saida_NZR_I.png}
                    \caption{comparação entre Emissor e receptor}
                    \label{fig:Saida_NZR_I}
                \end{center}
            \end{figure}
        \subsection{Manchester}
            Nessa codificação o bit 0 é representado por uma queda de tensão no meio do bit e o bit 1 e representado por um aumento da tensão no meio do bit. Os gráficos a seguir mostram a comparação entre entrada, sinal filtrado e saída:
            \begin{figure}[H]
                \begin{multicols}{2}
                    \begin{figure}[H]
                        \begin{center}
                            \includegraphics[scale=0.45]{Manchester_1.png}
                            \caption{1/2 harmônica}
                            \label{fig:Manchester_1}
                        \end{center}
                    \end{figure}
                    \begin{figure}[H]
                        \begin{center}
                            \includegraphics[scale=0.45]{Manchester_3.png}
                            \caption{3 harmônica}
                            \label{fig:Manchester_3}
                        \end{center}
                    \end{figure}
                    \begin{figure}[H]
                        \begin{center}
                            \includegraphics[scale=0.45]{Manchester_2.png}
                            \caption{1 harmônica}
                            \label{fig:Manchester_2}
                        \end{center}
                    \end{figure}
                    \begin{figure}[H]
                        \begin{center}
                            \includegraphics[scale=0.45]{Manchester_4.png}
                            \caption{5 harmônica}
                            \label{fig:Manchester_4}
                        \end{center}
                    \end{figure}
                \end{multicols}
            \end{figure}
            \begin{figure}[H]
                \begin{multicols}{2}
                    \begin{figure}[H]
                        \begin{center}
                            \includegraphics[scale=0.45]{Manchester_5.png}
                            \caption{7 harmônica}
                            \label{fig:Manchester_5}
                        \end{center}
                    \end{figure}
                    \begin{figure}[H]
                        \begin{center}
                            \includegraphics[scale=0.45]{Manchester_6.png}
                            \caption{9 harmônica}
                            \label{fig:Manchester_6}
                        \end{center}
                    \end{figure}
                \end{multicols}
                %\captionsetup{figure = 2}
                \caption{Sinal de entrada vs. Sinal filtrado vs. Sinal de saída de Manchester}
                \label{fig:Manchester}
            \end{figure}
            Nesse caso percebemos que a comunicação foi eficaz, também, na quinta harmônica e usamos o mesmo tipo de implementação que nos NZR-I logo os tempos do bit tão são exatos pois trabalha com a amostragem dos bit e os seleciona baseado na diferença entre o elemento 25 e 75 do mesmo bit assim a diferença entre esses ponto indicara se é o bit 0 ou o bit 1.
            
            Abaixo temos a comparação entre a entrada e a saída. A variável "data out" é uma matriz de 6 linhas e 13 colunas pois as linhas são as saídas de cada harmônica:
            \begin{figure}[H]
                \begin{center}
                    \includegraphics[scale=0.5]{Saida_Manchester.png}
                    \caption{comparação entre Emissor e receptor}
                    \label{fig:Saida_Manchester}
                \end{center}
            \end{figure}
            É nítido o fato de que a comunicação só foi eficaz a partir da quita harmônica.
        \subsection{Manchester Diferencial}
            Nessa codificação o bit 0 é por um transição no inicio do bit e uma inversão no meio enquanto o bit 1 é representado por manter a tensão do bit anterior e uma inversão no meio do bit. Os gráficos a seguir mostram a comparação entre entrada, sinal filtrado e saída:
            \begin{figure}[H]
                \begin{multicols}{2}
                    \begin{figure}[H]
                        \begin{center}
                            \includegraphics[scale=0.45]{Manchester_diferencial_1.png}
                            \caption{1/2 harmônica}
                            \label{fig:Manchester_diferencial_1}
                        \end{center}
                    \end{figure}
                    \begin{figure}[H]
                        \begin{center}
                            \includegraphics[scale=0.45]{Manchester_diferencial_3.png}
                            \caption{3 harmônica}
                            \label{fig:Manchester_diferencial_3}
                        \end{center}
                    \end{figure}
                    \begin{figure}[H]
                        \begin{center}
                            \includegraphics[scale=0.45]{Manchester_diferencial_5.png}
                            \caption{7 harmônica}
                            \label{fig:Manchester_diferencial_5}
                        \end{center}
                    \end{figure}
                    \begin{figure}[H]
                        \begin{center}
                            \includegraphics[scale=0.45]{Manchester_diferencial_2.png}
                            \caption{1 harmônica}
                            \label{fig:Manchester_diferencial_2}
                        \end{center}
                    \end{figure}
                
                    \begin{figure}[H]
                        \begin{center}
                            \includegraphics[scale=0.45]{Manchester_diferencial_4.png}
                            \caption{5 harmônica}
                            \label{fig:Manchester_diferencial_4}
                        \end{center}
                    \end{figure}
                    \begin{figure}[H]
                        \begin{center}
                            \includegraphics[scale=0.45]{Manchester_diferencial_6.png}
                            \caption{9 harmônica}
                            \label{fig:Manchester_diferencial_6}
                        \end{center}
                    \end{figure}
                \end{multicols}
                %\captionsetup{figure = 2}
                \caption{Sinal de entrada vs. Sinal filtrado vs. Sinal de saída de Manchester Diferencial}
                \label{fig:Manchester_diferencial}
            \end{figure}
            Nesse caso percebemos que a comunicação foi eficaz apenas na 7 harmônica e usamos o mesmo tipo de implementação que nos NZR-I e Manchester logo os tempos do bit tão são exatos. Ele trabalha analisando o bit anterior e fazendo ou não a transição no inicio do bit e fazendo a transição no meio dele.
            \begin{figure}[H]
                \begin{center}
                    \includegraphics[scale=0.45]{Saida_Manchester_diferencial.png}
                    \caption{comparação entre Emissor e receptor}
                    \label{fig:Saida_Manchester_diferencial}
                \end{center}
            \end{figure}
            Analisando os gráficos e o "data ou" é perceptível que a comunicação foi eficaz  na sétima harmônica.
    \section{Parte 4}
        Nessa parte aplicamos um ruido branco no código da parte 1 e 2. Usamos a função "unifrnd" para fazer a simulação do ruido. Analisando o código podemos perceber que a saída é definida se um ponto é maior ou meno que 2.5, caso maior é 1 caso menor é 0, logo se o ruido tiver uma interferência maior que 2.5 ele ira atrapalhar na comunicação. No gráfico abaixo temos o sinal filtrado somado ao retorno da função "unifrnd" com parâmetro de -3 até 3:
         \begin{figure}[H]
            \begin{center}
                \includegraphics[scale=0.45]{ruido_grafico.png}
                \caption{Emissor receptor com ruido}
                \label{fig:ruido_grafico}
            \end{center}
        \end{figure}
        Percebe-se que o sinal em geral tem o padrão da entrada mas o ruido pode afetou de forma desastrosa a comunicação. Esse código tinha uma comunicação quase eficaz a partir do primeiro harmônico e podemos perceber na imagem abaixo que o ruido arruinou a saída de quase todas os sinais, é questão de sorte a saída dar certo.
        \begin{figure}[H]
            \begin{center}
                \includegraphics[scale=0.45]{ruido_saida.png}
                \caption{Saída com ruido}
                \label{fig:ruido_saida}
            \end{center}
        \end{figure}
    \section{Conclusão}
        Podemos concluir que o uso do maior harmônico torna a comunicação mas eficaz e que dependo da implementação podemos resolver problemas relativamente simples mas que atrapalhariam a comunicação. Em nenhum dos códigos usados a cima foi feita uma media dos valores do sinal filtrado para obter o bit correspondente, é provável que ao fazer isso o ruido não atrapalhasse tanto quanto foi mostrado.
\end{document}